\documentclass{article}
\usepackage[margin=1in]{geometry}
\usepackage[utf8]{inputenc}
\usepackage{enumerate}


% for keeping track of the current article
\newcounter{articleNumber}
% for keeping track of the current section in the article
\newcounter{sectionNumber}

% Create a new article, using roman numerals. Reset the section counter
\newcommand{\article}[1]{\setcounter{sectionNumber}{0}\stepcounter{articleNumber}\bigskip{\LARGE\textbf{{\\Article \Roman{articleNumber}: #1}}}}
% Create a new section, have good spacing between sections, and have them align to the left
\newcommand{\sect}[1]{\stepcounter{sectionNumber}{\raggedright{\hfill\break\Large Section \Roman{sectionNumber}. #1}\\}}
% Create a signature line for the given position
\newcommand{\signature}[1]{\bigskip\bigskip\bigskip{\raggedright\makebox[2.5in]{\hrulefill}\\ #1}}

% Here we go!
\begin{document}

% Center the title; custom title formatting. Maybe should be commands?
\begin{center}
    {\Huge\textbf{The Constitution of the\\[.15cm]Upsilon Pi Epsilon Association}}

    {\LARGE The International Honors Society for the\\Computing and Information Disciplines}
\end{center}

\article{Name}

\sect{}

The formal name of this organization is \textbf{Upsilon Pi Epsilon - Beta Chapter of Connecticut}, hereinafter referred to as ``UPE''.

\article{Purpose}

\sect{}

The purpose of UPE shall be to promote high scholarship and original investigation in the several branches of the Computing and Information Disciplines.
UPE will work toward its goal by holding events, including a semesterly induction ceremony, performing community service to enrich the computer science program at UConn, and inviting guest speakers to discuss topics in computer science and related fields.

\article{Membership Composition}

\sect{}

Any person may attend UPE meetings.
However, to be a member of UPE, one must be inducted into the group.
Eligibility requirements are described in the International constitution above.
Membership implies paying membership dues.

\article{Organization Officers}

\sect{Duties}

\begin{enumerate}[(a)]
    \item Chief Organization Officer (COO). Spokesperson of the group that regularly interacts with other student organizations and University officials.
    \item Chief Financial Officer (CFO). This person will be primarily responsible for the organization's finances.
    \item Events Coordinator (EVC). The duties of the events coordinator are to facilitate event planning for the organization.
    \item Secretary. The secretary is responsible for taking minutes at meetings and distributing them to officers the following week.
\end{enumerate}

\sect{Officer election process}

Organization officers will be selected by a nomination and election process.
Officers shall be elected before the end of the Spring term and will serve a term of one year.
The time between elections and the new term shall serve as a transition period for new officers.

\pagebreak
\article{Organization Decision Making Model}

\sect{}

As a general rule, UPE shall use majority vote among officers to make its decisions.

\article{Meetings}

\sect{Frequency}

The day of the week and time of meetings will be chosen via majority vote among members.
Meetings will be held once a month on the chosen day and time.

\sect{Structure}

The meeting will begin by being called to order by the Chief Operating Officer.
Minutes will be taken by the Secretary, and distributed to the organization's officers the following week.
The meeting will be ended by the Chief Operating Officer.

\article{Organization Advisor}

\sect{}

A full time faculty member from the Storrs University of Connecticut campus is eligible for the faculty advisor position.
The faculty advisor will be chosen by the officers and their term will last indefinitely, with an annual reappointment.

\sect{}

The advisor's duties shall include

\begin{enumerate}[(a)]
    \item Meeting with the organization officers on a regular basis.
    \item Attending organization meetings and activities.
\end{enumerate}

\sect{}

The advisor shall not have voting rights

\article{Organization Funds}

\sect{}

Means of acquiring funds will include collection of dues, application to USG for funding, and implementation of fundraising events throughout the academic year.

\article{Amending the Constitution}

\sect{}

This constitution may be amended by a two-thirds vote of present members.
A quorum of officers (defined as 50\% of officers plus one) must be present to vote on a proposed amendment.
Upon receiving the two-thirds vote in favor of a constitutional amendment, the organization advisor, as well as the Involvement Office, will be informed of the amendment (by the Chief Operating Officer).

\pagebreak
\article{Beneficiary Addendum}

\sect{}

In the event that this organization's account remains inactive for 12 consecutive months, the following beneficiary will receive the balance of the organization's funds:

Beneficiary Name: UCONN ACM

Beneficiary Address: 371 Fairfield Way, Unit 4155, Storrs, CT, 06269

Beneficiary Contact: Alexander Shvartsman, aas@cse.uconn.edu

Phone: (860) 486-2672

\article{Enabling Clause}

\sect{}

This constitution was voted on and put into effect on Monday, April $14^{\textnormal{th}}$, 2014

\signature{President}

\signature{Vice President}

\signature{Treasurer}

\signature{Secretary}

\end{document}
